\documentclass{article}
\usepackage[english]{babel}
\usepackage{amsmath,amssymb,enumerate,hyperref}
\usepackage[tikz]{mdframed}

%%%%%%%%%% Start TeXmacs macros
\newcommand{\infixand}{\text{ and }}
\newcommand{\nin}{\not\in}
\newcommand{\nobracket}{}
\newcommand{\nosymbol}{}
\newcommand{\textdots}{...}
\newcommand{\tmem}[1]{{\em #1\/}}
\newcommand{\tmemail}[1]{\\ \textit{Email:} \texttt{#1}}
\newcommand{\tmop}[1]{\ensuremath{\operatorname{#1}}}
\newcommand{\tmrsub}[1]{\ensuremath{_{\textrm{#1}}}}
\newcommand{\tmstrong}[1]{\textbf{#1}}
\newcommand{\tmsubtitle}[1]{\thanks{\textit{Subtitle:} #1}}
\newcommand{\tmtextbf}[1]{\text{{\bfseries{#1}}}}
\newcommand{\tmtextit}[1]{\text{{\itshape{#1}}}}
\newcommand{\tmtextrm}[1]{\text{{\rmfamily{#1}}}}
\newcommand{\tmtexttt}[1]{\text{{\ttfamily{#1}}}}
\newcommand{\tmverbatim}[1]{\text{{\ttfamily{#1}}}}
\newenvironment{enumeratealpha}{\begin{enumerate}[a{\textup{)}}] }{\end{enumerate}}
\newenvironment{tmparmod}[3]{\begin{list}{}{\setlength{\topsep}{0pt}\setlength{\leftmargin}{#1}\setlength{\rightmargin}{#2}\setlength{\parindent}{#3}\setlength{\listparindent}{\parindent}\setlength{\itemindent}{\parindent}\setlength{\parsep}{\parskip}} \item[]}{\end{list}}
\mdfsetup{linecolor=black,linewidth=0.5pt,skipabove=0.5em,skipbelow=0.5em,hidealllines=true,innerleftmargin=0pt,innerrightmargin=0pt,innertopmargin=0pt,innerbottommargin=0pt}
\newtheorem{proposition}{Proposition}
\newmdenv[topline=true,bottomline=true,innertopmargin=1ex,innerbottommargin=1ex]{tmbothlined}
%%%%%%%%%% End TeXmacs macros

\begin{document}

\title{
  Enumeration of Combined BIBDs
  \tmsubtitle{}
}

\author{
  Andrei Ivanov
  \tmemail{{\tmem{andrey.ivanov@gmail.com}}}
}

\maketitle

\begin{abstract}
  A BIBD (Balanced Incomplete Block Design) $D = (V, \mathbb{B}) \tmop{is}
  \tmop{called} \embold{\text{{\tmem{combined}}}}$, if $\tmop{for} \tmop{some}
  \hspace{2}  \linebreak n \geqslant {\nobreak}  2 \tmop{the} \tmop{set}
  \tmop{of} \tmop{blocks} \mathbb{B} \tmop{is} \text{an a} \tmop{disjunctive}
  \tmop{association} \tmop{of} \{ \mathbb{B}_i \} \tmop{such} \tmop{that}
  \tmop{every} \linebreak D_i = \left( V, {\nobreak} \mathbb{B}_i \right)
  \tmop{is} \text{a} \tmop{BIBD} . \text{ In this paper we will describe the
  algorithm of enumeration of combined BIBDs and some of the results, obtained
  with it.}_{\nosymbol}$
  
  \ 
\end{abstract}

\section{Introduction}

Let $V$ be a set of the elements and \ensuremath{\mathbb{B}} is a collection
of some (not necessary different) subsets of $V.$ We will denote by $v = |V|$
and $b = |\mathbb{B}|$ the cardinalities of these sets. Usually, the elements
$B_j \in \mathbb{B}, 1 \leqslant j \leqslant b$ are called the blocks and the
pair $D$=($V$, $\mathbb{B}$) is called a (combinatorial)
{\tmem{{\tmstrong{block design}}}} or an
{\tmem{{\tmstrong{{\tmem{{\tmem{i{\tmem{{\tmem{ncidence
system}}}}}}}}{\tmem{}}}}}}.

\

\begin{tmparmod}{0pt}{0pt}{0tab}%
  For a block design $D$=($V$, $\mathbb{B}$) the number of blocks that have
  the same set of elements as the block $B_{\nosymbol} \in \mathbb{B}
  \tmop{will} \tmop{be} \tmop{called}$ {\tmstrong{{\tmem{the multiplicity}}}}
  of the that block:
\end{tmparmod}
\[ \mu (B_{\nosymbol}) = | \{ \nobracket B_i \in \mathbb{B}: B_i =
   B_{\nosymbol} \} | . \nobracket \]
\begin{tmparmod}{0pt}{0pt}{0tab}%
  Two block designs $D_i$=($V_i, \mathbb{B}_i$), $i = 1, 2$ are said to be
  {\tmstrong{{\tmem{isomorphic}}}} if $| V_1 | = | V_2 |$ and there exists a
  bijection function $\mathfrak{I}: V_1 \rightarrow V_2$ such that for
  $\tmop{any} \tmop{element} x \in V_1 \tmop{and} \tmop{any} \tmop{block} B_1
  \in \mathbb{B}_1$ of size $k_1 = | B_1 |$ which contains $x$, the element
  $\mathfrak{I} (x) \tmop{belongs} \tmop{exactly} \tmop{to} \mu (B_1)
  \tmop{blocks} B_2 \in \mathbb{B}_2$ of the same size $k_1$.\quad
  \[ \{ \nobracket \left\{ \mathfrak{I} (x) : x \in B_1 \text{\}} : B_1 \in
     \mathbb{B}_1 \text{\}}_{\nosymbol} \mu \left( {B_1}_{\nosymbol} \right) =
     \mu (B_2) \right\} =\mathbb{B}_{2.} \]
  In other words, if we rename every element {\tmem{x}} by
  $\mathfrak{I}$({\tmem{x}}), then the collection of blocks $\mathbb{B}_1$ is
  transformed into $\mathbb{B}_2$.The bijection $\mathfrak{I}$ is called an
  {\tmstrong{{\tmem{isomorphism}}}}.
  
  {\tmstrong{{\tmstrong{}}}}
\end{tmparmod}

\begin{tmparmod}{0pt}{0pt}{0tab}%
  Let \ $D$=($V$, $\mathbb{B}$) is a block design with $v = | V |
  \tmop{elements} \infixand b = | \mathbb{B} | \text{}$ blocks. The
  {\tmstrong{{\tmem{{\tmstrong{incidence matrix}}}}{\tmem{}}}} of this block
  design is the~$v \times b$ matrix $M (D)$ whose entries are defined as
\end{tmparmod}
\begin{equation}
  m_{i j} = \left\{\begin{array}{l}
    1, \tmop{if} x_i \in B_j \qquad\\
    0, \tmop{otherwise} .
  \end{array}\right.
\end{equation}
\begin{tmparmod}{0pt}{0pt}{0tab}%
  \
  
  It is easy to prove that two block designs $D_i$=($V_i, \mathbb{B}_i$), $i
  = 1, 2$, are isomorphic if and only if there are such permutation matrices
  $P \infixand Q, \tmop{acting}, \tmop{respectively}, \tmop{on} \tmop{the}
  \tmop{rows} \tmop{of} \infixand \tmop{columns} \tmop{of} \tmop{incidence}
  \tmop{matrices} M (D_i) \tmop{that} \tmop{the} \tmop{following} \tmop{holds}
  :$
  \begin{equation}
    P \cdot M (D_1) = M (D_2) \cdot Q.
  \end{equation}
\end{tmparmod}

\begin{tmparmod}{0pt}{0pt}{0tab}%
  Let $D_i = \left( V, \mathbb{B}_i \right)$, $i = 1, 2, \tmop{be}
  \tmop{two}$block designs defined on the same set of elements$V$ and
  $M_i$=$M_i (D_i) \tmop{are} \tmop{corresponding} \tmop{incidence}
  \tmop{matrices}$. By horizontally joining (concatenating) these matrices we
  can construct $| V | \times \left( \left| \mathbb{B}_1 \right| + \left|
  \mathbb{B}_2 \right| \right)$ matrix $M (D) = [M_1, M_2]$, $\tmop{which}
  \tmop{will} \tmop{be} \text{ an incidence } \tmop{matrix} $of some block
  design $D_{} = \left( V, \mathbb{B}_1 +\mathbb{B}_2 \right)$.
  
  \
  
  The block design (resp., incidence matrix) constructed by using such
  horizontal concatenation of the insidence matrices of the smaller block
  designs will be called
  {\tmem{{\tmem{{\tmem{{\tmstrong{combined{\tmem{}}}}}}{\tmstrong{}}}}}} block
  design (resp., incidence matrix).
\end{tmparmod}

\

\begin{tmparmod}{0pt}{0pt}{0tab}%
  The block design $D$=($V$, $\mathbb{B}$) is called a{\tmem{
  {\tmstrong{}}}}{\tmstrong{{\tmem{{\tmstrong{}}balanced incomplete block
  design}}}} (BIBD), if
\end{tmparmod}

{\nopagebreak}(a) each block $B_j \in \mathbb{B}$ contains the same number of
elements {\tmem{k}};

(b) each element $v_i \in V$ belongs to the same number of blocks $r$;

(c) any two of distinct elements $v_i, v_j \in V$ appear together in the same
number of blocks $\lambda$.

\

\begin{tmparmod}{0pt}{0pt}{0tab}%
  These five parameters ($v, b, r, k, \lambda$) satisfy following two
  equations:
\end{tmparmod}
\[ v r = {} b k, \]
\[ r (k - 1) = \lambda (v - 1), \]
which allow to express $r \infixand b$ as
\begin{equation}
  r = \lambda \frac{v - 1}{k - 1}, \text{{\hspace{3em}}} b = \lambda \frac{v
  (v - 1)}{k (k - 1)}
\end{equation}
Much more information regarding the BIBDs and related structures could be
found in [1].

\

\begin{tmparmod}{0pt}{0pt}{0tab}%
  Suppose, we do have two (not nessesary different or non-isomorfic) BIBD's:
  D\tmrsub{$i$}= (V,\ensuremath{\mathbb{B}}\tmrsub{$i$}), $i = 1, 2$ $,$with
  parameters $(v, k, \lambda_i)$. It means that not only all these designs are
  defined on the same set of elements $V, \tmop{but} \tmop{also} \tmop{that} |
  X | = | Y | = k, \tmop{for} \tmop{any} \tmop{two} \tmop{block} X \in
  \mathbb{B}_1, Y \in \mathbb{B}_2$.
\end{tmparmod}



\begin{tmparmod}{0pt}{0pt}{0tab}%
  Let's choose some permutation matrices $P_1 \infixand P_2$ acting
  respectively on the rows of incidence matrices $M_i = M  (D_i) .$It is easy
  to check that the combined incidence matrix $M = [P_1 \cdot M_1, P_2 \cdot
  M_2]$ satisfies the following conditions:
  \begin{enumeratealpha}
    \item each column of $M$ contains the exactly {\tmem{k}} 1's;
    
    \item each row of $M$ contains exactly $r = r_1 + r_2 = (\lambda_1 +
    \lambda_2)  (v - 1) / (k - 1)$ \ 1's
    
    \item two distinct rows of $M$ contain both 1's in exactly $\lambda =
    \lambda_1 + \lambda_2 \quad \tmop{columns} .$
  \end{enumeratealpha}
  Therefore, the combined incidence matrix $M = [P_1 \cdot M_1, P_2 \cdot
  M_2]$ corresponds to some combined BIBD (CBIBD).
  
  \
  
  The concatenation of incidence matrices of BIBD's having the same
  parameters ($v, k$) is probably the easiest way to construct bigger BIBD's
  from smaller ones. Using
  \begin{enumeratealpha}
    \item different pairs $(\tilde{P}_1, \tilde{P}_2),$generally speaking, we
    can construct pairwise non-isomorphic BIBDs with the same parameters;
    
    \item two, three or more ``initial'' BIBDs as units, we can construct
    BIBDs that are composed of two, three or more BIBD's of smaller sizes.
  \end{enumeratealpha}
\end{tmparmod}

\paragraph{}

\begin{tmparmod}{0pt}{0pt}{0tab}%
  A typical problem in combinatorics is the construction of a complete set of
  pairwise non-isomorphic combinatorial objects with given properties. This is
  commonly referred to as the problem of enumerating such objects. Our goal:
  to construct an algorithm that would allow one to enumerate CBIBDs for $n >
  {\nobreak} 1$ and given set of parameters $(v, k, \{ \lambda_1, \ldots,
  \lambda_n \}$), where $(v, k, \lambda_i) \tmop{are} \tmop{the}
  \tmop{parameters} \tmop{of} \tmop{BIBD} \tmop{for} 1 \leqslant {\nobreak} i
  \leqslant {\nobreak} n.$
\end{tmparmod}

\section{Representation of the Combined BIBDs}

We will follow the methodology described in [2] and [5], where it is
recommended to build a complete list of so-called
{\tmstrong{{\tmem{canonical}} }}objects for solving a particular combinatorial
enumeration problem. Here is a brief description of this technique.

\

\begin{tmparmod}{0pt}{0pt}{0tab}%
  Suppose we need to enumerate the combinatorial objects with some set of
  specified properties. Let \ensuremath{\mathcal{A}} be a set of all such
  objects and $G = {\nobreak} G (\mathcal{A})$ be a group of transformations
  applied to objects from \ensuremath{\mathcal{A}} such that $g (\alpha) \in
  \mathcal{A}$for $\forall \alpha \in {\nobreak} \mathcal{A} \infixand \forall
  g \in G. \text{ Usually such a transformation group G is called
  {\tmem{{\tmstrong{the isomorphism group}}}}} .$
\end{tmparmod}

\

\begin{tmparmod}{0pt}{0pt}{0tab}%
  In most cases, the isomorphism group $G$ is somehow related to the
  renumbering of some components of objects from \ensuremath{\mathcal{A}},
  $\tmop{but} \tmop{it} \tmop{can} \tmop{also} \tmop{include}
  \tmop{transformations} \tmop{of} \tmop{some} \tmop{other} \tmop{types} .$
\end{tmparmod}

\

\begin{tmparmod}{0pt}{0pt}{0tab}%
  Examples of transformations of the first type are the renumbering of
  vertices of ordinary graphs or elements and blocks of incidence systems,
  etc. Other examples of transformations that can also appear in $G$ when
  enumerating combinatorial objects of the corresponding classes are:
  replacing edges with non-edges (and vice versa) in regular graphs of degree
  $k$ with $v = 2 k + 1$ vertices, or replacing incidences with non-incidences
  (and vice versa) in BIBDs with $v = 2 k$ elements, or even the transposition
  of the incidence matrices of symmetric BIBDs.
\end{tmparmod}

\

\begin{tmparmod}{0pt}{0pt}{0tab}%
  Suppose that each object $\alpha${\in}\ensuremath{\mathcal{A}} has some
  matrix representation $M (\alpha)$. It could be an adjacency matrix of a
  graph, an incidence matrix of a block design, an $n \times n$ matrix
  representing \text{a} Latin square etc. For matrices $M (\alpha), \text{ } M
  (\beta)$ \ we can determine the binary precedence relation, for example, by
  lexicographic comparison of their rows. We will say that $\tmop{the}
  \tmop{matrix} M (\alpha) \tmop{is}
  \tmem{\text{{\tmstrong{greater{\tmem{}}}}{\tmstrong{}}}} \tmop{than}
  \tmop{the} \tmop{matrix} M (\beta)$ if for some $n \geqslant 0$ their first$
  \tmem{n}$ rows are identical, but (n+1)-th row of $M (\alpha) \tmop{is}
  \tmop{lexicographically} \tmop{greater} \tmop{than}$(n+1)-th row of $M
  (\beta) . \tmop{This} \tmop{fact} \tmop{will} \tmop{be} \tmop{expressed}
  \tmop{as} :$
  \[ M (\alpha) \succ M (\beta) \]
\end{tmparmod}

\begin{tmparmod}{0pt}{0pt}{0tab}%
  {\tmem{{\tmem{{\tmstrong{Definition 2.1: }} {\tmstrong{}}}}{\tmstrong{}}}}A
  combinatorial object $\alpha${\in}\ensuremath{\mathcal{A}} (and its matrix
  representation $M (\alpha)$) will be called {\tmstrong{{\tmem{canonical}}}}
  with respect to the isomorphism group {\tmem{G}}, if for $\forall g \in G$:
\end{tmparmod}
\[ M (\alpha) \succcurlyeq M (g (\alpha)) . \]
By definition, the canonicity or non-canonicity of the combinatorial oject
$\alpha \in$\ensuremath{\mathcal{A}} could be detrnined by its matrix
representation $M (\alpha) \infixand \tmop{the}$group
$G$(\ensuremath{\mathcal{A}}). In many cases, this fact greatly simplifies the
construction of all existing pairwise non-isomorphic combinatorial objects of
the class \ensuremath{\mathcal{A}}, since there is no need to maintain lists
of already constructed objects and check for isomorphism for each newly
constructed object.

\

\begin{tmparmod}{0pt}{0pt}{0tab}%
  {\tmem{{\tmem{{\tmstrong{Definition 2.2: }} {\tmstrong{}}}}{\tmstrong{}}}}A
  balanced incomplete block design $D = (V, \mathbb{B})$ will be called
  {\tmem{{\tmstrong{combined of rank n}} {\tmstrong{}}}}if the set of block
  $\tmem{} \mathbb{B}$ can be divided into $n \geqslant 2$ disjoint subsets
  $(\tmop{components})  \{ \mathbb{B}_i \}$, in such a way that $D_i = (V,
  {\nobreak} \text{{\nobreak}} \mathbb{B}_i) \tmop{will} \tmop{be} \text{ a}
  \tmop{balanced} \tmop{incomplete} \tmop{block} \tmop{design} \tmop{for}
  \tmop{each} 1 \leqslant i \leqslant n$.
\end{tmparmod}

\

\begin{tmparmod}{0pt}{0pt}{0tab}%
  Since $D_i = (V, {\nobreak} \text{{\nobreak}} \mathbb{B}_i) \tmop{is}
  \text{a} \tmop{BIBD} \tmop{with} \tmop{parameters} (v, k, \lambda {}_i) $for
  each $0 \leqslant i \leqslant n$, it's very logical to use {\noindent}
  \begin{equation}
    (v, k, \{ \lambda {}_1, \lambda {}_2, \ldots, \lambda {}_n \})
  \end{equation}
  as the parameters of combined BIBD $D = (V, \mathbb{B})$. 
\end{tmparmod}

\

\begin{tmparmod}{0pt}{0pt}{0tab}%
  We will represent a combined BIBD {\tmem{D}} by the the pair $(\mathfrak{m},
  M (D)) , \tmop{where} \mathfrak{m}$=($\mathfrak{m}_j$) is a vector of length
  \textbar$\mathbb{B} |, \tmop{with} \tmop{integer} \tmop{coordinates}
  \nobracket$indicating the occurrence of corresponding block in
  $\mathbb{B}_i, M (D) \tmop{is} \tmop{the} \tmop{incidence} \tmop{matrix}
  \tmop{of} D . \text{}$
\end{tmparmod}

\

\begin{tmparmod}{0pt}{0pt}{0tab}%
  In what follows, we will represent this pair $(m, M (D))$ in one matrix:
\end{tmparmod}
\begin{equation}
  \mathbb{M} (D) = \left[ \underset{M (D)}{\overset{m }{}} \right] \quad
  \tmop{where} m = (m_j) \infixand m_j = i, \tmop{if} B_j \in \mathbb{B}_i
  \tmop{for} 1 \leqslant j \leqslant | \mathbb{B} |
\end{equation}


\begin{tmparmod}{0pt}{0pt}{0tab}%
  with \textbar V\textbar +1 rows and \textbar\ensuremath{\mathbb{B}}\textbar 
  columns, and we will call (2.1) {\tmem{{\tmstrong{the}}{\tmstrong{}}}}
  {\tmem{{\tmstrong{matrix representation}}}} of {\tmem{D}}.
\end{tmparmod}

\

\begin{tmparmod}{0pt}{0pt}{0tab}%
  To formulate the CBIBD enumeration problem precisely, we also need
\end{tmparmod}

a) determine the group of isomorphisms acting on these objects;

b) clarify some parameters of the enumerated CBIBDs.

\

\begin{tmparmod}{0pt}{0pt}{0tab}%
  Let us first focus on (a) and consider the direct product
  $\mathfrak{G}=$\tmverbatim{$\mathcal{V} \times \mathcal{B}$}of symmetric
  groups \tmverbatim{$\mathcal{V}$}and $\mathcal{B}$, acting on the rows and
  columns of matrix $M (D)$, respectively. In accordance with the Definition
  2.1, the matrix $\mathbb{M} (D) \tmop{will} \tmop{be} \tmop{called}
  \tmop{canonical} \tmop{with} \tmop{respect} \tmop{to} \tmop{the}$isomorphism
  group $\mathfrak{G}$, if 
\end{tmparmod}
\[ \mathbb{M} (D) \succcurlyeq \mathfrak{g} (\nobracket \mathbb{M} (D)),
   \tmop{for} \nobracket \tmop{any} \mathfrak{g} \in \mathfrak{G}. \]
\begin{proposition}
  If $\mathbb{M} (D)  \text{defined by {\tmem{(2.1)}}}  \tmem{}$is a canonical
  matrix of CBIBD of rank n then the coordinates of the vector $m = (m_j)$ are
  \begin{equation}
    (m_j) = (n, \ldots, n, n - 1, \ldots, n - 1, n - 2, \ldots n - 2, \ldots,
    2, \ldots, 2, 1, \ldots 1)
  \end{equation}
  which means that all blocks of$\normalsize{\LARGE{}} $BIBD $D_i = (V,
  {\nobreak} \text{{\nobreak}} \mathbb{B}_i)$ are represented in $M (D) 
  \text{by the columns with consecutive indices{\tmem{}}} $ $\text{for any $1
  \leqslant i \leqslant n.$}$
\end{proposition}

\begin{flushleft}
  \begin{tmparmod}{0pt}{0pt}{0tab}%
    \begin{tmparmod}{0pt}{0pt}{0tab}%
      {\tmem{Proof:}} Let's consider the set of indices$I_n =$\{$i_1, i_2,
      \textdots,${\tmem{i$_{| \mathbb{B}_n |}$}}\} of the columns of
      $\mathbb{M} (D) $which correspond to all blocks from ${\nobreak}
      \text{{\nobreak}} \mathbb{B}_n$. By definition of$\tmop{vector} m =
      (m_j)$, the equality $m_l = n \tmop{is} \tmop{true} \tmop{for} \forall l
      \text{} \in \text{{\noindent}} {\nobreak} I_n \text{{\nobreak}.}$
    \end{tmparmod}
  \end{tmparmod}
\end{flushleft}

\

\begin{tmparmod}{0pt}{0pt}{0tab}%
  Since $n \tmop{is} \text{a{\tmem{}}} \tmop{maximal} \tmop{value}
  \tmop{among} \tmop{all}$coordinates $(m_j)$, $1 \leqslant j \leqslant |
  \mathbb{B} | \infixand \tmop{the}$matrix $\mathbb{M} (D) \tmop{is}
  \tmop{canonical}, \text{any index {\tmem{l}}{\tmem{$\in$I}}$_n$}$ must
  precede any index $x$ with a smaller value $m_x < n.$ Otherwise, by
  rearranging the columns{\tmem{ l}} and {\tmem{x}} of the matrix
  \ensuremath{\mathbb{M}}({\tmem{D}}) we could get the matrix
  $\tilde{\mathbb{M}}$({\tmem{D}}) with the first row
  $\widetilde{m}${\succ}{\nobreak}{\tmem{m}}, which contradicts our assumption
  about the canonicity of \ensuremath{\mathbb{M}}({\tmem{D}}). \
  
  \
  
  Using the same arguments for \ ${\nobreak} \text{{\nobreak}} \mathbb{B}_{n
  - 1}, {\nobreak} \text{{\nobreak}} \mathbb{B}_{n - 2}, \ldots, {\nobreak}
  \text{{\nobreak}} \mathbb{B}_1, \tmop{we}$get (2.3).
  
  \ 
\end{tmparmod}

\begin{tmparmod}{0pt}{0pt}{0tab}%
  Denote by $b {}_i$=\textbar${\nobreak} \text{{\nobreak}}
  \mathbb{B}_i$\textbar, the number of blocks of $i$-th component $D_i = (V,
  {\nobreak} \text{{\nobreak}} \mathbb{B}_i), 1 \leqslant i \leqslant n
  \tmop{of} \tmop{CBIBD} .$
  
  \
  
  \paragraph{Proposition 2.2.}{\tmem{If}} $\mathbb{M} (D) 
  \text{{\tmem{defined by}} {\tmem{{\tmem{(2.1)}}}}}  \tmem{}${\tmem{is a
  canonical matrix of CBIBD of rank }} $n,$ {\tmem{the}}{\tmem{n\quad}}$b {}_s
  \text{} \geqslant \text{} b {}_t $ {\tmem{for any}} $1 \leqslant s \leqslant
  \text{$t \leqslant n.$}$
  
  \
  
  \begin{tmparmod}{0pt}{0pt}{0tab}%
    \text{\text{\text{\text{\text{{\tmem{Proof:}} By definition, $b
    {}_i$=\textbar${\nobreak} \text{{\nobreak}} \mathbb{B}_i$, $1 \leqslant i
    \leqslant n$is equal to the number of occurrences of $i$ in the vector $m
    \text{=} (m_j) . \tmop{According} \tmop{to}  \tmop{Proposition} 2.1,
    \text{when $\mathbb{M} (D) \tmop{is} \tmop{canonical},$all coordinates }
    \tmop{of} \tmop{this} \tmop{vector} \tmop{satisfy} (2.2), \tmop{which}
    \tmop{can} \tmop{be} \tmop{rewritten} \tmop{as} \text{} \text{} $}}}}}
    \[ (m_j) = \left( [b {}_1 \ast n], \ldots, [b {}_p \ast (n - p + 1),
       \ldots, [b {}_n \ast 1]) \text{}, \right. \]
    $\tmop{where} [b {}_a \ast (n - a + 1)] \tmop{is} \text{a} \tmop{vector}
    \tmop{with} \tmop{all} b {}_a \tmop{coordinates} \tmop{equal} \tmop{to} (n
    - a + 1) .$
    
    \
    
    $\tmop{If} \tmop{the} \tmop{statement} \tmop{of}$Proposition 2.2 is
    false, then for some $s$ and $t$, \ $1 \leqslant s \leqslant \text{$t
    \leqslant n,$}$ $b {}_s \text{} < \text{} b {}_t$ we do have
    \[ (m_j) = ([b {}_1 \ast n], \ldots, [b {}_s \ast (n - s + 1)], \ldots, [b
       {}_t \ast (n - t + 1)], \ldots ., [b {}_n \ast 1]) \text{} . \]
    \begin{flushleft}
      Keeping the indices of the components $\{\mathbb{B} {}_i \}, 1 \leqslant
      i \leqslant n, i \neq s, t$ the same and changing the indices of the
      components \ensuremath{\mathbb{B}}${}_s$ and
      \ensuremath{\mathbb{B}}${}_t$ by $t$ and$s$, respectively, we will see
      that the vector
    \end{flushleft}
    \[ (\widetilde{m_j}) = ([b {}_1 \ast n], \ldots, [b {}_t \ast (n - s +
       1)], \ldots, [b {}_s \ast (n - t + 1)], \ldots ., [b {}_n \ast 1])
       \text{} \]
    is greater then $(m_j)$. For both of these vectors the first $(n - s)
    \tmop{blocks} \tmop{of} \tmop{their} \tmop{coordinates}, \tmop{equal}
    \tmop{to} \text{} \text{{\noindent} } n, (n - 1), \ldots, (s + 1),
    \tmop{respectively},$will be the same and the block $\tmop{of} \tmop{the}
    \tmop{vector} (\widetilde{m_j})$ with the coordinates equal to $s
    \tmop{will} \tmop{be} \tmop{longer}, \tmop{since}$ $b {}_t \text{} >
    \text{} b {}_s$. Thus, we have obtained the relation $(\widetilde{m_j})
    \succ (m_j)$ which contradicts our assumption about the canonicity of the
    matrix \ $\mathbb{M} (D) .$
    
    \ 
  \end{tmparmod}
  
  A direct consequence of Proposition 2.2 is the following
  
  
  
  {\tmstrong{Proposition 2.3.}}{\tmem{ For the canonical CBIBD of rank n with
  parameters}} ($v, k, (\lambda {}_1, \lambda {}_2, \ldots \lambda {}_n)$):
  \[ \lambda {}_i \geqslant \lambda {}_j  \quad
     \tmem{\text{\text{for{\tmem{}}}{\tmem{}}}} \quad \forall (i, j) : 1
     \leqslant i \leqslant j \leqslant n. \]
  {\tmem{Proof:}} According to (1.3)
  \[ \text{} b_i = \lambda_i  \frac{v_i (v_i - 1)}{k_i (k_i - 1)} = \text{}
     \lambda_i  \frac{v (v - 1)}{k (k - 1)} \quad \tmop{for} 1 \leqslant i
     \leqslant n. \]
  As we could see, all $\lambda {}_i \tmop{are} \tmop{proportional} \tmop{to}
  \text{} b_i$ with the same proportionality factor: $k (k - 1) / \left( v
  \left( v - {\nobreak} 1 \right) \right) . \text{ But according to
  Proposition 2.2 $b_i \geqslant b_j $for $\forall (i, j) : 1 \leqslant i
  \leqslant j \leqslant n . \tmop{Therefore}, \lambda {}_i \geqslant \lambda
  {}_j$.}$
  
  \section{Isomorphisms of Combined BIBDs}
  
  If {\tmem{D }}is a canonical CBIBD with parameters \ ($v, k, \{ \lambda
  {}_1, \lambda {}_2, \ldots \lambda {}_n \}$), then, \ according to
  Propositions 2.1 and 2.2, its matrix representation must have the following
  form:
  \begin{eqnarray*}
    \mathbb{M} (D) = \left[ \underset{\tmem{B_1 \qquad } \hspace{3em}
    \tmem{B_2 \hspace{5em} \ldots \quad} \quad \tmem{B_{\tmem{\tmem{n}}}
    }}{\overset{n \ldots n \qquad \quad (n - 1) \ldots (n - 1) \quad \ldots
    \qquad 1. . . 1 }{}} \right] &  & \text{}  (3.1)
  \end{eqnarray*}
  
  
  where {\tmem{B\tmrsub{i}{\tmem{}}}} is a balanced incomplete block design
  with parameters \ ($v, k, \lambda {}_i$) for $1 \leqslant i \leqslant n.$
  Let's describe the group of transformations $G (D) $which will be used for
  canonicity check of {\tmem{D}}.
  
  \
  
  First of all, as in the case of BIBD, it is natural to include in this
  group the symmetric group
  {\tmem{$S$\tmrsub{v}}}=$S$\tmrsub{{\tmem{v}}}({\tmem{V}}), acting on the set
  of elements {\tmem{V}}.
  
  \
  
  Second, \begin{itemize}
    of course we should consider
  \end{itemize} the direct product of the groups: \
  \[ \tmem{\text{S} (\mathbb{B}) =} S {}_{b {}_1} \times S {}_{b {}_2} \times
     \ldots \times S {}_{b {}_n}, \]
  where $S {}_{b {}_i}$ = $S {}_{b {}_i}$($B {}_i$) $1 \leqslant i \leqslant
  n,$ is a symmetrical group, acting on the blocks of corresponding BIBD $B
  {}_i .$
  
  
  
  Finally, we also need to take into account the obvious combinatorial
  symmetry of the CBIBD, components that have the same parameter {\lambda}. To
  do this, we define by $\Lambda {}_j \tmop{the} \tmop{following} \tmop{set}
  \tmop{of} \tmop{components} : \text{}$
  \[ \Lambda {}_j = \left\{ B {}_i \left| \large{} \lambda {}_i \right. = j
     \right\} \tmop{for} j \in \{ \lambda {}_i | 1 \leqslant i \leqslant n
     \nobracket \} . \]
  direct product of the groups:
  \[ \text{{\tmem{$S (\Lambda) =$}}} S {}_{| \Lambda \nobracket {}_1 |
     \nobracket} \times S {}_{| \Lambda \nobracket {}_2 | \nobracket} \times
     \ldots \times S {}_{| \Lambda \nobracket {}_{j {}_{\max}} | \nobracket},
  \]
  where $S {}_{| \Lambda \nobracket {}_j | \nobracket} = S {}_{| \Lambda
  \nobracket {}_j | \nobracket} (\Lambda {}_j), 1 \leqslant j \leqslant j
  {}_{\max} = | \{ \Lambda {}_j \} |$, is a symmetrical group, acting on the
  corresponding components of CBIBD.
  
  \
  
  Thus, the full set of isomorphisms of the CBIBD will be defined as a direct
  product:
  \[ \Gamma (D) = S {}_v (V) \times \tmem{\text{S} (\mathbb{B}) \times S
     (\Lambda)} \]
  of the groups $S {}_v (V), \tmem{\text{S} (\mathbb{B}), S (\Lambda)}$ acting
  on the corresponding CBIBD elements.
  
  \
  
  The following example of the CBIBD with parameters (6, 3, \{4, 2, 2\})
  illustrates the importance of considering $S (\Lambda)$. For that design $D
  {}_1$ the matrix
  \begin{eqnarray*}
    \mathbb{M} (D {}_1) = \left[ \underset{\textit{\tmem{B_1 \quad} \quad
    \tmem{B_2}  \qquad \tmem{B_{\tmem{\tmem{3}}} }}}{\overset{3 \ldots 3 \quad
    2 \ldots 2 \quad 1. . . 1 }{}} \right] &  & 
  \end{eqnarray*}
  has the form:
  
  \tmtexttt{{\hspace{9em}}33333333333333333333 2222222222 1111111111
  
  {\hspace{9em}}11111111110000000000 1111100000 1111100000
  
  {\hspace{9em}}11110000001111110000 1100011100 1100011100
  
  {\hspace{9em}}11001100001100001111 {\underline{0011011010}}
  {\underline{1010010011}}\quad
  
  {\hspace{9em}}00101011000011101110 1000110011 0101001011
  
  {\hspace{9em}}00010010111010011101 0010101101 0001110110
  
  {\hspace{9em}}00000101110101110011 0101000111 0010101101}
  
  \
  
  It is easy to see that by changing the places of $B {}_2$ and $B {}_3$ we
  will get combined BIBD $D {}_2$ which is isomorphic to $D {}_1, \tmop{but}
  \tmop{its} \tmop{representation} \tmop{matrix}$
  \[ \begin{array}{l}
       \mathbb{M} (D {}_2) = \left[ \underset{\tmem{B_1 \quad \quad}
       \tmem{B_3} \quad \quad \tmem{B_{\tmem{\tmem{2}}} }}{\overset{3 \ldots 3
       \quad 2 \ldots 2 \quad 1. . . 1 }{}} \right]
     \end{array} \]
  is lexicographically greater than the matrix $\mathbb{M} (D {}_1) :$
  $\mathbb{M} (D {}_2) \succ \mathbb{M} (D {}_1) .$ No other isomorphic
  transformation of $D {}_1 \tmop{from}$group
  $S$\tmrsub{{\tmem{v}}}({\tmem{V}})$\times \text{S} (\mathbb{B}) \tmop{will}
  \tmop{give} \tmop{us} \tmop{this} \tmop{result} .$
  
  \section{Enumeration results of some Combined BIBDs}
  
  For the enumeration of CBIBDs with parameters \ ($v, k, \{ \lambda {}_1,
  \lambda {}_2, \ldots \lambda {}_n \}$), we slightly modified the algorithm
  described in [3]. The most significant changes affected:
  \begin{enumerate}
    \item computation and storage of solutions for construction of
    {\tmem{j}}-th rows of matrices $\{ B {}_i \}, 1 \leqslant {\nobreak} i
    \leqslant {\nobreak} n ;$
    
    \item checking the canonicity of fully and partially constructed matrices
    $\mathbb{M} (D), \tmop{defined} \tmop{by} (2.2), \tmop{including}$the use
    of $S (\Lambda)$.
  \end{enumerate}
  The program that implements this algorithm and some results of its use for
  enumeration of CBIBDs, BIBDs, t-designs, and semisymmetric graphs can be
  downloaded from [4].
  
  \
  
  The following table contains some enumeration results of CBIBDs on $v
  \tmop{elements} \tmop{for} 6 \leqslant {\nobreak} v \leqslant 13 :
  \text{{\tmstrong{\\
  \\
  Table 1}}{\tmstrong{}}} .$
  
  {\tmem{\qquad$(v, k, \{ \lambda {}_1, \lambda {}_2, \ldots \})$ \ \ \
  \qquad\qquad Total \#: \quad Simple \#: \ \qquad  Run Time (sec):}}
  
  \begin{tmbothlined}
    \tmtexttt{\tmtexttt{{\hspace{3em}}(6, 3, \{2, 2\}) \ \ \ \ \ \ \ \ \
    {\hspace{4em}}2 \ \ \ \ \ \ \ \ \ \ 2 \ \ \ \ \ \ \ \ \ \ \ \ 0.00
    
    {\hspace{3em}}(6, 3, \{4, 2\}) \ \ \ \ \ \ \ \ {\hspace{4em}}12 \ \ \ \ \
    \ \ \ \ \ 1 \ \ \ \ \ \ \ \ \ \ \ \ 0.02
    
    {\hspace{3em}}(6, 3, \{4, 4\}) \ \ \ \ \ \ \ \ {\hspace{4em}}24 \ \ \ \ \
    \ \ \ \ \ 1 \ \ \ \ \ \ \ \ \ \ \ \ 0.02
    
    {\hspace{3em}}(6, 3, \{2, 2, 2\}) \ \ \ \ \ \ {\hspace{4em}}5 \ \ \ \ \ \
    \ \ \ \ 5 \ \ \ \ \ \ \ \ \ \ \ \ 0.00
    
    {\hspace{3em}}(6, 3, \{4, 2, 2\}){\hspace{7em}}43 \ \ \ \ \ \ \ \ \ \ 2 \
    \ \ \ \ \ \ \ \ \ \ \ 0.02
    
    {\hspace{3em}}(6, 3, \{4, 4, 2\}) \ \ \ \ {\hspace{4em}}131 \ \ \ \ \ \ \
    \ \ \ 1 \ \ \ \ \ \ \ \ \ \ \ \ 0.04
    
    {\hspace{3em}}(6, 3, \{2, 2, 2, 2\}) \ \ {\hspace{4em}}10 \ \ \ \ \ \ \ \
    \ 10 \ \ \ \ \ \ \ \ \ \ \ \ 0.02
    
    {\hspace{3em}}(6, 3, \{4, 2, 2, 2\}) \ {\hspace{4em}}124 \ \ \ \ \ \ \ \ \
    \ 5 \ \ \ \ \ \ \ \ \ \ \ \ 0.05
    
    {\hspace{3em}}(6, 3, \{4, 4, 2, 2\}) \ {\hspace{4em}}604 \ \ \ \ \ \ \ \ \
    \ 2 \ \ \ \ \ \ \ \ \ \ \ \ 0.20
    
    {\hspace{3em}}(6, 3, \{4, 4, 4, 2\}) {\hspace{4em}}1722 \ \ \ \ \ \ \ \ \
    \ 1 \ \ \ \ \ \ \ \ \ \ \ \ 1.05
    
    {\hspace{3em}}(6, 3, \{4, 4, 4, 4\}) {\hspace{4em}}2490 \ \ \ \ \ \ \ \ \
    \ 1 \ \ \ \ \ \ \ \ \ \ \ \ 6.43
    
    {\hspace{3em}}(7, 3, \{1, 1\}) \ \ \ \quad \ \ \ \ \ {\hspace{3em}}2 \ \ \
    \ \ \ \ \ \ \ 2 \ \ \ \ \ \ \ \ \ \ \ \ 0.00
    
    {\hspace{3em}}(7, 3, \{2, 1\}) \ \ \ \ \ \quad \ \ {\hspace{3em}}16 \ \ \
    \ \ \ \ \ \ \ 3 \ \ \ \ \ \ \ \ \ \ \ \ 0.01
    
    {\hspace{3em}}(7, 3, \{1, 1, 1\}) \ \ \ \ \quad \ {\hspace{3em}}5 \ \ \ \
    \ \ \ \ \ \ 5 \ \ \ \ \ \ \ \ \ \ \ \ 0.00
    
    {\hspace{3em}}(7, 3, \{2, 1, 1\}) \ \ \ \ \ {\hspace{4em}}88 \ \ \ \ \ \ \
    \ \ 20 \ \ \ \ \ \ \ \ \ \ \ \ 0.02
    
    {\hspace{3em}}(7, 3, \{2, 2, 1\}) \ \ \ \ {\hspace{4em}}607 \ \ \ \ \ \ \
    \ \ 31 \ \ \ \ \ \ \ \ \ \ \ \ 0.12
    
    {\hspace{3em}}(7, 3, \{2, 2, 2\}) \ \ \ {\hspace{4em}}2319 \ \ \ \ \ \ \ \
    \ 33 \ \ \ \ \ \ \ \ \ \ \ \ 1.38
    
    {\hspace{3em}}(8, 4, \{3, 3\}) \ \ \ {\hspace{6em}}601 \ \ \ \ \ \ \ \ 601
    \ \ \ \ \ \ \ \ \ \ \ \ 0.18
    
    {\hspace{3em}}(8, 4, \{6, 3\}) \ \ {\hspace{4em}}10648675 \ \ \ \ \ 776249
    \ \ \ \ \ \ \ \ \ 5:18.60
    
    {\hspace{3em}}(8, 4, \{3, 3, 3\}) {\hspace{4em}}1044344 \ \ \ \ 1044344 \
    \ \ \ \ \ \ \ \ 3:23.01
    
    {\hspace{3em}}(9, 3, \{1, 1\}) \ \ \ \ \ \ \ \ \ {\hspace{4em}}3 \ \ \ \ \
    \ \ \ \ \ 3 \ \ \ \ \ \ \ \ \ \ \ \ 0.03
    
    {\hspace{3em}}(9, 3, \{2, 1\}) \ \ \ \ \ \ {\hspace{4em}}8039 \ \ \ \ \ \
    \ 3514 \ \ \ \ \ \ \ \ \ \ \ \ 0.57
    
    {\hspace{3em}}(9, 3, \{2, 2\}) \ \ {\hspace{4em}}16407107 \ \ \ \ 4068169
    \ \ \ \ \ \ \ \ 14:34.27
    
    {\hspace{3em}}(9, 3, \{3, 1\}) \ \ {\hspace{4em}}15574414 \ \ \ \ \ 208362
    \ \ \ \ \ \ \ \ \ 8:09.02
    
    {\hspace{3em}}(9, 3, \{1, 1, 1\}) \ \ \ \ {\hspace{3em}} \quad 61 \ \ \ \
    \ \ \ \ \ 61 \ \ \ \ \ \ \ \ \ \ \ \ 0.16
    
    {\hspace{3em}}(9, 3, \{2, 1, 1\}) {\hspace{4em}}2768644 \ \ \ \ 1156282 \
    \ \ \ \ \ \ \ \ 2:51.29
    
    {\hspace{3em}}(9, 3, \{1, 1, 1, 1\}){\hspace{4em}}22727 \ \ \ \ \ \ 22727
    \ \ \ \ \ \ \ \ \ \ \ 26.06
    
    {\hspace{3em}}(9, 3, \{2, 1, 1, 1\})\qquad 672239828 \ \ 300916861 \
    \qquad 31:13:43.60}
    
    {\hspace{3em}}(10,4, \{2, 2\})\quad \ \ \ {\hspace{3em}} \quad 7613 \ \ \
    \ \ \ \ 7613 \ \ \ \ \ \ \ \ \quad \ 4.79
    
    {\hspace{3em}}(10,4, \{2, 2, 2\})\quad \ \ \ {\hspace{3em}}7613 \ \ \ \ \
    \ \ 7613 \ \ \ \ \ \ {\hspace{3em}}4.79
    
    {\hspace{3em}}(10,4, \{2, 2\})\quad \ \ \ {\hspace{3em}} \quad 7613 \ \ \
    \ \ \ \ 7613 \ \ \ \ \ \ \ \ \quad \ 4.79
    
    {\hspace{3em}}(11,5, \{2, 2\})\quad \ \ \ {\hspace{3em}} \quad 7613 \ \ \
    \ \ \ \ 7613 \ \ \ \ \ \ \ \ \quad \ 4.79
    
    {\hspace{3em}}(11,5, \{2, 2, 2\})\quad \ \ \ {\hspace{3em}}7613 \ \ \ \ \
    \ \ 7613 \ \ \ \ \ \ \ \ \quad \ 4.79
    
    {\hspace{3em}}(13,3, \{1, 1\})\quad \ \ \ {\hspace{3em}} \quad 7613 \ \ \
    \ \ \ \ 7613 \ \ \ \ \ \ \ \ \quad \ 4.79
    
    {\hspace{3em}}(13,4, \{1, 1\})\quad \ \ \ {\hspace{3em}} \quad 7613 \ \ \
    \ \ \ \ 7613 \ \ \ \ \ \ \ \ \quad \ 4.79
    
    \ }
  \end{tmbothlined}
  
  In the {\tmem{{\tmstrong{}}}} column ``{\tmem{Total}} \#'' of this table, we
  list the number of pairwise non-isomorphic combined BIBDs, and the column \
  {\tmem{{\tmem{``{\tmem{Simple}} \#''{\tmem{}}}}}} contains the number of
  non-isomorphic CBIBDs, all of which have no replicated blocks.
  
  \
  
  For any set of CBIBD parameters from {\tmstrong{Table 1}} in [4] one can
  find detailed information about the automorphism groups of constructed sets
  of designs in the following form:
  
  \
  
  \tmtexttt{{\tmstrong{\tmtextrm{Table 2.}}}
  
  CBIBD(7, 3, \{2, 1, 1\})
  
  \
  
  \ \ \ \ \ \ \ \textbar Aut($D$)\textbar \ \ \ \ \ \ \ \ \ \ \ Nd: \ \ \ \
  \ \ \ \ \ \ Ns: \ \ Ndt: \ \ Nst:
  
  \ \ \
  \_\_\_\_\_\_\_\_\_\_\_\_\_\_\_\_\_\_\_\_\_\_\_\_\_\_\_\_\_\_\_\_\_\_\_\_\_\_\_\_\_\_\_\_\_\_\_\_\_\_\_\_\_\_\_\_\_\_\_\_\_
  
  \ \ \ \ \ \ \ \ \ \ \ \ 1 \ \ \ \ \ \ \ \ \ \ \ \ \ \ 19 \ \ \ \ \ \ \ \ \
  \ \ 8 \ \ \ \ \ 0 \ \ \ \ \ 0
  
  \ \ \ \ \ \ \ \ \ \ \ \ 1*2 \ \ \ \ \ \ \ \ \ \ \ \ \ 7 \ \ \ \ \ \ \ \ \ \
  \ 1 \ \ \ \ \ 0 \ \ \ \ \ 0
  
  \ \ \ \ \ \ \ \ \ \ \ \ 2 \ \ \ \ \ \ \ \ \ \ \ \ \ \ \ 5 \ \ \ \ \ \ \ \ \
  \ \ 0 \ \ \ \ \ 0 \ \ \ \ \ 0
  
  \ \ \ \ \ \ \ \ \ \ \ \ 2*2 \ \ \ \ \ \ \ \ \ \ \ \ \ 3 \ \ \ \ \ \ \ \ \ \
  \ 0 \ \ \ \ \ 0 \ \ \ \ \ 0
  
  \ \ \ \ \ \ \ \ \ \ \ \ 3 \ \ \ \ \ \ \ \ \ \ \ \ \ \ 17 \ \ \ \ \ \ \ \ \
  \ \ 7 \ \ \ \ \ 0 \ \ \ \ \ 0
  
  \ \ \ \ \ \ \ \ \ \ \ \ 3*2 \ \ \ \ \ \ \ \ \ \ \ \ \ 7 \ \ \ \ \ \ \ \ \ \
  \ 2 \ \ \ \ \ 0 \ \ \ \ \ 0
  
  \ \ \ \ \ \ \ \ \ \ \ \ 4 \ \ \ \ \ \ \ \ \ \ \ \ \ \ \ 4 \ \ \ \ \ \ \ \ \
  \ \ 0 \ \ \ \ \ 0 \ \ \ \ \ 0
  
  \ \ \ \ \ \ \ \ \ \ \ \ 4*2 \ \ \ \ \ \ \ \ \ \ \ \ \ 6 \ \ \ \ \ \ \ \ \ \
  \ 0 \ \ \ \ \ 0 \ \ \ \ \ 0
  
  \ \ \ \ \ \ \ \ \ \ \ \ 6 \ \ \ \ \ \ \ \ \ \ \ \ \ \ \ 1 \ \ \ \ \ \ \ \ \
  \ \ 0 \ \ \ \ \ 0 \ \ \ \ \ 0
  
  \ \ \ \ \ \ \ \ \ \ \ \ 6*2 \ \ \ \ \ \ \ \ \ \ \ \ \ 1 \ \ \ \ \ \ \ \ \ \
  \ 0 \ \ \ \ \ 0 \ \ \ \ \ 0
  
  \ \ \ \ \ \ \ \ \ \ \ \ 8 \ \ \ \ \ \ \ \ \ \ \ \ \ \ \ 1 \ \ \ \ \ \ \ \ \
  \ \ 0 \ \ \ \ \ 0 \ \ \ \ \ 0
  
  \ \ \ \ \ \ \ \ \ \ \ \ 8*2 \ \ \ \ \ \ \ \ \ \ \ \ \ 1 \ \ \ \ \ \ \ \ \ \
  \ 0 \ \ \ \ \ 0 \ \ \ \ \ 0
  
  \ \ \ \ \ \ \ \ \ \ \ 12 \ \ \ \ \ \ \ \ \ \ \ \ \ \ \ 2 \ \ \ \ \ \ \ \ \
  \ \ 0 \ \ \ \ \ 0 \ \ \ \ \ 0
  
  \ \ \ \ \ \ \ \ \ \ \ 12*2 \ \ \ \ \ \ \ \ \ \ \ \ \ 4 \ \ \ \ \ \ \ \ \ \
  \ 0 \ \ \ \ \ 0 \ \ \ \ \ 0
  
  \ \ \ \ \ \ \ \ \ \ \ 21 \ \ \ \ \ \ \ \ \ \ \ \ \ \ \ 1 \ \ \ \ \ \ \ \ \
  \ \ 0 \ \ \ \ \ 1 \ \ \ \ \ 0
  
  \ \ \ \ \ \ \ \ \ \ \ 21*2 \ \ \ \ \ \ \ \ \ \ \ \ \ 3 \ \ \ \ \ \ \ \ \ \
  \ 2 \ \ \ \ \ 3 \ \ \ \ \ 2
  
  \ \ \ \ \ \ \ \ \ \ \ 24 \ \ \ \ \ \ \ \ \ \ \ \ \ \ \ 1 \ \ \ \ \ \ \ \ \
  \ \ 0 \ \ \ \ \ 0 \ \ \ \ \ 0
  
  \ \ \ \ \ \ \ \ \ \ \ 24*2 \ \ \ \ \ \ \ \ \ \ \ \ \ 4 \ \ \ \ \ \ \ \ \ \
  \ 0 \ \ \ \ \ 0 \ \ \ \ \ 0
  
  \ \ \ \ \ \ \ \ \ \ 168*2 \ \ \ \ \ \ \ \ \ \ \ \ \ 1 \ \ \ \ \ \ \ \ \ \ \
  0 \ \ \ \ \ 1 \ \ \ \ \ 0
  
  \ \ \
  \_\_\_\_\_\_\_\_\_\_\_\_\_\_\_\_\_\_\_\_\_\_\_\_\_\_\_\_\_\_\_\_\_\_\_\_\_\_\_\_\_\_\_\_\_\_\_\_\_\_\_\_\_\_\_\_\_\_\_\_\_
  
  \ \ \ \ \ \ \ Total: \ \ \ \ \ \ \ \ \ \ \ \ \ \ 88 \ \ \ \ \ \ \ \ \ \ 20
  \ \ \ \ \ 5 \ \ \ \ \ 2}
  
  \tmtexttt{\
  
  \ \ \ \ \ \ \ 88 matrices were constructed in \ 0.02 sec,
  
  \ \ \ \ \ \ \ \ 5 of them are transitive on the rows.
  
  \ \ \ \ \ \ \ 20 matrices have no replicated blocks,
  
  \ \ \ \ \ \ \ \ 2 of them are transitive on the rows.
  
  \ }
  
  If information about \textbar Aut($D$)\textbar  is represented as a number
  $N$, this means that $N$ is the order of the group acting on the elements of
  the design, and there are no nontrivial automorphisms acting on the
  components of the corresponding CBIBDs. If it is represented as
  {\tmem{N*M}}, then$N$ and$M$ are the orders of automorphism groups acting
  respectively on the elements and components of the corresponding CBIBDs. 
\end{tmparmod}

\

\section{Master-BIBDs of Combined BIBDs}

As we already know, any CBIBD with parameters \ ($v, k, \{ \lambda {}_1,
\lambda {}_2, \ldots \lambda {}_n \}$) is also an ``ordinary'' BIBD with
parameters \ ($v, k, \lambda {}_1 + \lambda {}_2 + \ldots + \lambda {}_n$). In
other words, if $D$ is a CBIBD represented by the matrix $\mathbb{M} (D)
\tmop{defined} \tmop{by} (3.1), \tmop{then}$
\begin{equation}
  \mathbb{B} (D) = [B {}_1, B {}_2, \ldots, B {}_n]
\end{equation}
is an incidence matrix of some BIBD. In what follows, we will call such BIBD
$\mathbb{B} (D) ${\tmstrong{a master BIBD}} of $D.$We think it would be very
interesting to investigate how often two or more non-isomorphic CBIBDs have
isomorphic master BIBDs. Or a similar, but somewhat differently worded
question: how often does BIBDs have different (non-isomorphic) representations
by CBIBDs?

\begin{tmparmod}{0pt}{0pt}{0tab}%
  To solve this problem, we have used the following approach.
\end{tmparmod}

\

\begin{tmparmod}{0pt}{0pt}{0tab}%
  \begin{tmparmod}{0pt}{0pt}{0tab}%
    We have created a database $\mathfrak{D}$for storing the canonical
    master-BIBDs. At the beginning of the enumeration process of CBIBDs with
    parameters \ ($v, k, \{ \lambda {}_1, \lambda {}_2, \ldots \lambda {}_n
    \}$, this database $\mathfrak{D}$is empty. When the program builds the
    next canonical CBIBD $D, \tmop{we} \tmop{use} (5.1) \tmop{to}
    \tmop{create} \tmop{the} \tmop{corresponding} \linebreak \tmop{master}
    {\nobreak} {\nobreak} \text{-{\nobreak}} \tmop{BIBD} \mathbb{B} (D)
    \infixand \tmop{define} \tmop{its} \tmop{canonical} \tmop{representation}
    \mathbb{C} (D) = \tmop{Canon} (\mathbb{B} (D)) .$Using \textbar
    Aut($\mathbb{C} (D)$)\textbar  and matrix 
  \end{tmparmod}$\mathbb{C} (D)$, respectively, as the primary and secondary
  keys of the database $\mathfrak{D}$, we are trying to find $\mathbb{C} (D)
  \tmop{in} \mathfrak{D}. \textit{\text{{\tmem{If such a search was not
  successful, we add the newly constructed canonical master-BIBD to our
  database $\mathfrak{D}$with the counter equal to 1. Otherwise, we just
  increase the counter of the corresponding master-BIBD.}}}}  \text{}$
\end{tmparmod}

\

\begin{tmparmod}{0pt}{0pt}{0tab}%
  The results of a master-BIBD search for a given set of combined BIBD
  parameters are presented in tables similar to the following one:
\end{tmparmod}

{\tmstrong{\\
Table 3.}}{\tmstrong{}}

\tmtexttt{CBIBD(7, 3, \{2, 1, 1\})

\

\textbar Aut(M)\textbar : {\tmem{Masters: \ CBIBDs: \ \ \ \ \ \ \
Distribution:}}

\_\_\_\_\_\_\_\_\_\_\_\_\_\_\_\_\_\_\_\_\_\_\_\_\_\_\_\_\_\_\_\_\_\_\_\_\_\_\_\_\_\_\_\_\_\_\_\_\_\_\_\_\_\_\_\_

\ \ \ \ \ 1 \ \ \ \ \ \ \ \ \ 1 \ \ \ \ \ \ \ 6 \ \ \ 6

\ \ \ \ \ 2 \ \ \ \ \ \ \ \ \ 6 \ \ \ \ \ \ 17 \ \ \ 2*2 + 3*3 + 4

\ \ \ \ \ 3 \ \ \ \ \ \ \ \ \ 5 \ \ \ \ \ \ 16 \ \ \ 1 + 3 + 3*4

\ \ \ \ \ 4 \ \ \ \ \ \ \ \ \ 3 \ \ \ \ \ \ \ 8 \ \ \ 2*2 + 4

\ \ \ \ \ 6 \ \ \ \ \ \ \ \ \ 4 \ \ \ \ \ \ 11 \ \ \ 1 + 2*3 + 4

\ \ \ \ \ 8 \ \ \ \ \ \ \ \ \ 1 \ \ \ \ \ \ \ 3 \ \ \ 3

\ \ \ \ 12 \ \ \ \ \ \ \ \ \ 3 \ \ \ \ \ \ \ 5 \ \ \ 1 + 2*2

\ \ \ \ 16 \ \ \ \ \ \ \ \ \ 2 \ \ \ \ \ \ \ 5 \ \ \ 2 + 3

\ \ \ \ 21 \ \ \ \ \ \ \ \ \ 1 \ \ \ \ \ \ \ 2 \ \ \ 2

\ \ \ \ 24 \ \ \ \ \ \ \ \ \ 4 \ \ \ \ \ \ 10 \ \ \ 1 + 2*2 + 5

\ \ \ \ 42 \ \ \ \ \ \ \ \ \ 1 \ \ \ \ \ \ \ 2 \ \ \ 2

\ \ \ \ 48 \ \ \ \ \ \ \ \ \ 1 \ \ \ \ \ \ \ 2 \ \ \ 2

\ \ \ 168 \ \ \ \ \ \ \ \ \ 1 \ \ \ \ \ \ \ 1 \ \ \ 1

\_\_\_\_\_\_\_\_\_\_\_\_\_\_\_\_\_\_\_\_\_\_\_\_\_\_\_\_\_\_\_\_\_\_\_\_\_\_\_\_\_\_\_\_\_\_\_\_\_\_\_\_\_\_\_\_

\ Total: \ \ \ \ \ \ \ 33 \ \ \ \ \ \ 88 \ \ \ MaxDecomp for master: 6}

\

\begin{tmparmod}{0pt}{0pt}{0tab}%
  In {\tmstrong{Table 3}}, the master-BIBDs corresponding to all existing
  pairwise non-isomorphic Combined BIBDs with parameters ($v, k, \{ \lambda
  {}_1, \lambda {}_2, \lambda {}_3 \}$) = (7,3,\{2,1,1\}) are grouped
  according to the orders of their automorphism groups.
  
  \
  
  In the {\tmstrong{{\tmem{Masters}}}} column, we indicate the total number
  of non-isomorphic master BIBDs with parameters ($v, k, \nobracket \lambda
  {}_1 + \lambda {}_2 + \lambda {}_3) =$(7,3,4) that have an automorphisms
  group of the corresponding order.
  
  \
  
  In the {\tmem{{\tmstrong{CBIBDs{\tmem{}}}}}} column, we indicate the total
  number of Combined BIBDs, each of which corresponds to some master with a
  given automorphism group order.
  
  \
  
  Finally, in the {\tmem{{\tmstrong{Distribution}}}} column, we indicate how
  the CBIBDs from the previous column are distributed among their masters.
  This information is showing as a
  \begin{equation}
    \underset{}{} \underset{i}{\sum} \underset{}{} p {}_i \ast n {}_i
  \end{equation}
  
\end{tmparmod}

\begin{tmparmod}{0pt}{0pt}{0tab}%
  where $p {}_i $is the number of master-BIBDs, each corresponding to exactly
  $n {}_i$ non-isomorphic CBIBDs. 
\end{tmparmod}

\begin{tmparmod}{0pt}{0pt}{0tab}%
  In this representation, we will order ($p {}_i n {}_i$) by increasing values
  of $n {}_i$ and we will not write $p {}_i \tmop{when} \tmop{it} \tmop{is}
  \tmop{equal} \tmop{to} 1.$
\end{tmparmod}

\

\begin{tmparmod}{0pt}{0pt}{0tab}%
  It is easy to see that by definition of the numbers $p {}_i,$ $n {}_i$ for
  any row of such table \ the sums $\underset{i}{\sum} \underset{}{} p {}_i$
  and $\underset{}{} \underset{i}{\sum} \underset{}{} p {}_i n {}_i $ are
  always equal to the numbers presented in columns
  {\tmem{{\tmstrong{Masters}}}} and {\tmem{{\tmstrong{CBIBDs}}}},
  respectively.
\end{tmparmod}

\

\begin{tmparmod}{0pt}{0pt}{0tab}%
  In the last row of such table we provide the information regarding the total
  numbers of non-isomorphic master-BIBs \ and CBIBDs and the number
  {\tmem{{\tmstrong{MaxDecomp for master}}}}, which is the $\max (n {}_i)$
  among all previous rows of that table.
\end{tmparmod}



\begin{tmparmod}{0pt}{0pt}{0tab}%
  As we can see from the first row of {\tmstrong{Table 3}}, there are 6
  pairwise non-isomorphic Combined BIBDs that correspond to the same
  master-BIBD with \textbar Aut(M)\textbar =1. In other words, that
  master-BIBD can be constructed in 6 different ways. \
  
  \
  
  From the third row of {\tmstrong{Table 3}}, we can see that 16 pairwise
  non-isomorphic Combined BIBDs correspond to 5 non-isomorphic master-BIBDs
  with \textbar Aut(M)\textbar  = 3,
  
  \quad a) one of which could be obtained in 1 way;
  
  \quad b) one of which could be obtained in 3 different ways;
  
  \quad c) three of which could be obtained in 4 different ways each.
  
  \
  
  Thus, out of 88 non-isomorphic Combined BIBDs mentioned in {\tmstrong{Table
  2}}, only 33 non-isomorphic master-BIBDs with parameters ($v, k, \nobracket
  \lambda ) =$(7,3,4) could be built, and one of them could be obtained in 6
  different ways (from 6 non-isomorphic CBIBDs with parameters \ ($v, k, \{
  \lambda {}_1, \lambda {}_2, \lambda {}_3 \}$) =  (7,3,\{2,1,1\}).
  
  \
  
  Just to show how many different non-isomorphic decompositions can be
  associated with a single master-BIBD, we provide the following: 
\end{tmparmod}

{\tmstrong{\\
Table 4.}}{\tmstrong{}}

\tmtexttt{CBIBD(6, 3, \{4, 4, 4, 4\})

\

\textbar Aut(M)\textbar : {\tmem{Masters: \ CBIBDs: \ \ \ \ \ \ \
Distribution:}}

\_\_\_\_\_\_\_\_\_\_\_\_\_\_\_\_\_\_\_\_\_\_\_\_\_\_\_\_\_\_\_\_\_\_\_\_\_\_\_\_\_\_\_\_\_\_\_\_\_\_\_\_\_\_\_\_\_\_\_\_\_\_\_\_\_\_\_\_\_\_\_\_\_\_\_\_\_\_\_\_\_\_\_\_\_\_\_

\ \ \ \ \ 1 \ \ \ \ \ \ \ \ \ 5 \ \ \ \ \ 105 \ \ \ 7 + 10 + 23 + 27 + 38

\ \ \ \ \ 2 \ \ \ \ \ \ \ \ 12 \ \ \ \ \ 495 \ \ \ 2 + 2*4 + 7 + 9 + 2*12 +
13 + 67 + 94 + 111 + 160

\ \ \ \ \ 3 \ \ \ \ \ \ \ \ 17 \ \ \ \ \ 597 \ \ \ 2 + 3*3 + 3*4 + 3*6 + 9 +
31 + 43 + 47 + 51 + 101 + 274

\ \ \ \ \ 4 \ \ \ \ \ \ \ \ \ 3 \ \ \ \ \ \ 75 \ \ \ 2 + 22 + 51

\ \ \ \ \ 6 \ \ \ \ \ \ \ \ 11 \ \ \ \ \ 450 \ \ \ 1 + 2*2 + 3 + 10 + 2*13 +
28 + 30 + 130 + 218

\ \ \ \ \ 8 \ \ \ \ \ \ \ \ \ 2 \ \ \ \ \ 135 \ \ \ 3 + 132

\ \ \ \ 12 \ \ \ \ \ \ \ \ 12 \ \ \ \ \ 174 \ \ \ 2*1 + 4*2 + 8 + 9 + 13 + 15
+ 57 + 62

\ \ \ \ 24 \ \ \ \ \ \ \ \ \ 8 \ \ \ \ \ 358 \ \ \ 1 + 3 + 7 + 10 + 43 + 45 +
120 + 129

\ \ \ \ 36 \ \ \ \ \ \ \ \ \ 1 \ \ \ \ \ \ 18 \ \ \ 18

\ \ \ \ 60 \ \ \ \ \ \ \ \ \ 4 \ \ \ \ \ \ 71 \ \ \ 1 + 2 + 13 + 55

\ \ \ 720 \ \ \ \ \ \ \ \ \ 1 \ \ \ \ \ \ 12 \ \ \ 12

\_\_\_\_\_\_\_\_\_\_\_\_\_\_\_\_\_\_\_\_\_\_\_\_\_\_\_\_\_\_\_\_\_\_\_\_\_\_\_\_\_\_\_\_\_\_\_\_\_\_\_\_\_\_\_\_\_\_\_\_\_\_\_\_\_\_\_\_\_\_\_\_\_\_\_\_\_\_\_\_\_\_\_\_\_\_\_

\ Total: \ \ \ \ \ \ \ 76 \ \ \ \ 2490 \ \ \ MaxDecomp for master: 274}

\

\begin{tmparmod}{0pt}{0pt}{0tab}%
  As we see here, each of the 76 existing BIBDs with parameters ($v, k,
  \lambda $)=(6,3,16) \ can be represented as some combination of 4 BIBDs with
  parameters \ ($v, k, \lambda $)=(6,3,4) and for one of them there are 274
  such non-isomorphic representations.
\end{tmparmod}

\

\begin{tmparmod}{0pt}{0pt}{0tab}%
  In [4] we provide information about master-BIBDs for 167 different sets of
  parameters of Combined BIBDs with$\text{ } 6 \leqslant v \leqslant 11$.
  Looking at these results, you may see that the number of non-isomorphic
  master-BIBDs for Combined BIBDs with parameters ($v, k, \{ \lambda {}_1,
  ..., \lambda {}_n \}$) is often too close or even coincides with the number
  of non-isomorphic ``ordinary'' BIBDs with parameters \ ($v, k, \lambda {}_1
  + ... + \lambda {}_n$). Therefore, we think that it would be interesting to
  take a closer look at BIBDs, which {\tmstrong{could not be represented}} as
  a combination of BIBDs of smaller sizes. 
\end{tmparmod}

\

\section{Quasi-Combined BIBDs}

\begin{tmparmod}{0pt}{0pt}{0tab}%
  \text{Let $\Lambda  = \{ \lambda \nobracket {}_i$\textbar$1 \leqslant i
  \leqslant n$\}, $n =$\textbar$\Lambda | \geqslant 2, \nobracket$ be such a
  set of $\tmop{not} \tmop{necessarily} \tmop{different} \lambda {}_i$ that
  $\tmop{the} \tmop{BIBDs}$ $D_i = {\nobreak} \left( V, {\nobreak}
  \mathbb{B}_i \right)$ with $\tmop{parameters} (v, k, \lambda {}_i)
  \tmop{exists} \tmop{for} 1 \leqslant i \leqslant n.$}
\end{tmparmod}

\

\begin{tmparmod}{0pt}{0pt}{0tab}%
  {\tmem{{\tmem{{\tmstrong{Definition 6.1: }} {\tmstrong{}}}}{\tmstrong{}}}}A
  BIBD $D_{} = \left( V, \mathbb{B} (\Lambda)_{} \right)$ with parameters ($v,
  k, \lambda = \lambda {}_1 + \lambda {}_2 + \ldots + \lambda {}_n$) will be
  called {\tmstrong{{\tmem{combined}}}} (respectively,
  $\text{\text{}{\tmem{{\tmstrong{quasi-combined}}}})
  \text{{\tmem{{\tmstrong{design of rank}} }}$n${\tmem{}}} }$in relation to
  $\Lambda  = \{ \lambda \nobracket {}_i$\textbar$1 \leqslant i \leqslant
  n$\}, $\tmop{if} \tmop{the} \tmop{set} \mathbb{B} \tmop{of} \tmop{its}
  \tmop{blocks} \tmop{could} (\tmop{respectively}, \tmop{could} \tmop{not})
  \tmop{be} \tmop{represented} \tmop{as}$
  \begin{equation}
    \mathbb{B} (\Lambda) = \{ \mathbb{B} \nobracket {}_i | \mathbb{B} {}_i
    \cap \mathbb{B} {}_j = \varnothing, 1 \leqslant i, j \leqslant n, i \neq
    j\},
  \end{equation}
  where (V,$\mathbb{B} {}_i$) is a BIBD with parameters $(v, k, \lambda {}_i)
  \tmop{for} \tmop{any} 1 \leqslant i \leqslant n.$
  
  \
  
  Given a set $\Lambda$ and one of its elements $\lambda {}_s \subset
  \Lambda, 1 \leqslant s \leqslant n,$we formally define the following two
  sets:
  \[ \[ \text{\text{\tmrsub{ }{\tmem{}}}{\tmem{}}} \Lambda_s^- =
        \tmem{\text{\text{\tmrsub{ }}} \Lambda_s^- {}_{\text{{\tmem{}}}}
        \left( \Lambda, \text{$\lambda_t$} \right) =} \{ \nobracket \Lambda  -
        \{ \lambda \nobracket {}_s, \lambda {}_t \}\} \cup \{ \lambda
        \nobracket {}_s + \lambda {}_t \}, \tmop{for} 1 \leqslant t \leqslant
        n, s \neq t \] \]
  \[ \Lambda_s^+ = \Lambda_s^+ \tmem{\left( \Lambda,
     \widetilde{\text{$\lambda$}} \right)}  \text{=\{$\Lambda - \lambda
     {}_s$\}$\cup \left\{ \tilde{\lambda} {}_{}, \text{$\lambda {}_s$-}
     \tilde{\lambda}  \right\}$ } \tmop{for} 1 \leqslant \tilde{\lambda}  <
     \lambda {}_s . \]
  It's easy to see that \textbar$\Lambda_s^- {}_{\text{{\tmem{}}}} \left(
  \Lambda, \text{$\lambda_t$)} \right| = | \Lambda | -$1 and
  \textbar$\Lambda_s^+ {}_{\text{{\tmem{}}}} \left( \Lambda,
  \text{$\lambda_t$} \right) | \nobracket = | \Lambda | +$1 since in the first
  case two elements
  
  $\lambda {}_s, \lambda {}_t \subset \Lambda $ are replaced by their sum
  $\lambda {}_s + \lambda {}_t$ and \ in second case one element $\lambda {}_s
  \subset \Lambda \tmop{is} \tmop{replaced} \tmop{by} \tmop{two} : \linebreak 
  \tilde{\lambda} {}_{}, \text{$\lambda {}_s$-} \tilde{\lambda}$.
  
  
  
  {\tmstrong{Proposition 6.1:}} If $D_{} = \left( V, \mathbb{B} (\Lambda)_{}
  \right)$ is a combined design of rank $n \geqslant 3$ in relation to the set
  \text{$\text{} \Lambda  = {\nobreak} \{ \lambda \nobracket {}_i$\textbar$1
  \leqslant i \leqslant n$}\}, than it is also a combined design of rank $(n -
  1)$ in relation to $\text{\text{the set\tmrsub{ }{\tmem{}}}{\tmem{}}}
  \text{\text{\tmrsub{ }{\tmem{}}}{\tmem{}}} \Lambda_s^- {}_{\text{{\tmem{}}}}
  \left( \Lambda, {\nobreak} \text{$\lambda_t$} \right) \linebreak \tmop{for}
  \tmop{any} 1 \leqslant s, t \leqslant n, s \neq t .$
  
  \
  
  {\tmstrong{Proof}}: Because of (6.1), for any $1 \leqslant s, t \leqslant
  n, s \neq t, \text{}$
  \[ (\mathbb{B} {}_s \cup \mathbb{B} {}_t) \cap \mathbb{B} {}_i
     \text{=$\varnothing$ for $1 \leqslant i \leqslant n, i \nin \{ s, t \}
     .$} \]
  It's obvious that when both $(V, \nobracket \mathbb{B} {}_s$), $(V,
  \nobracket \mathbb{B} {}_t$) are BIBDs and $\mathbb{B} {}_s \cap \mathbb{B}
  {}_s \tmop{is} \tmop{empty}, $ (V,$\mathbb{B} {}_s \cup \mathbb{B} {}_t$) is
  also a BIBD. Thus, by Definition 6.1, \ $\text{$D
  \text{\tmrsub{{\tmem{}}}{\tmem{}}}$}_{} {}_{\text{}} = \left( V, \mathbb{B}
  \left( \text{\text{}} \Lambda_s^- {}_{\text{{\tmem{}}}} \left( \Lambda,
  \text{$\lambda_t$} \right) \right)_{} \right)$ is a combined design of rank
  $n - 1 .$
  
  \
  
  {\tmstrong{Proposition 6.2:}} If $D_{} = \left( V, \mathbb{B} (\Lambda)_{}
  \right)$ is quasi-combined design of rank $n \geqslant 2$ in relation to the
  set \text{$\text{} \Lambda  = \{ \lambda \nobracket {}_i$\textbar$1
  \leqslant i \leqslant n$}\}, then it is also a quasi-combined design of rank
  $n + 1 \tmop{in} \tmop{relation} \tmop{to} \tmop{the} \tmop{set}$
  $\Lambda_s^+ \tmem{\left( \Lambda, \widetilde{\text{$\lambda$}} \right)
  \tmop{for} \tmop{any} 1 \leqslant s \leqslant n \infixand \tmop{any} 1
  \leqslant \tilde{\lambda}  < \lambda {}_s .}$
  
  \
  
  \text{{\tmstrong{Proof:}} Assume that the assertion of Proposition 6.2 is
  false. This means that $D_{} = (V, \mathbb{B} ( \Lambda_s^+  \tmem{}$)) is
  combined design of rank $(n + 1)$ in relation to the set $\Lambda_s^+ =
  \Lambda_s^+ \tmem{\left( \Lambda, \widetilde{\text{$\lambda$}} \right)} $
  for some $1 \leqslant s \leqslant n \tmop{some} 1 \leqslant \tilde{\lambda} 
  < {\nobreak} \lambda {}_s$
  
  \
  
  \text{Let the indices of the elements $\tilde{\lambda}  \infixand \lambda
  {}_s - \tilde{\lambda}  \tmop{in} \Lambda_s^+ \tmop{be} p \infixand r,
  \tmop{respectively} .$By Proposition 6.1, $D_{} \tmop{is} \tmop{also}
  \text{a} \tmop{combined} \tmop{design} \tmop{of} \tmop{rank} n \tmop{in}
  \tmop{relation} \tmop{to} \tmop{the} \tmop{set} \Lambda_p^-
  {}_{\text{{\tmem{}}}} \left( \Lambda_s^+ \tmem{}, {\nobreak}
  \text{$\lambda_r$} \right) \tmop{which} \tmop{coincides} \tmop{with}
  \Lambda$}according}
  
  to the definition of the indices $p$ and $r$ and the numbers $\lambda
  \text{${}_p$} = \tilde{\lambda}$, $\lambda \text{${}_r$=} \lambda {}_s -
  \tilde{\lambda}$ from $\Lambda_s^+ .$
  \[ \  \]
  (V,$\mathbb{B} {}_s \cup \mathbb{B} {}_t$)
  
  \
  
  \
  
  quasi-combined with respect to the set $\Lambda, \tmop{if} \tmop{the}
  \tmop{set} B \tmop{of} \tmop{its} \tmop{blocks} \tmop{could} \tmop{not}
  \tmop{be} \tmop{divided} \tmop{into} \tmop{subsets} \{ B {}_i \} \quad,
  \tmop{such} \tmop{that}$if A combinatorial object
  $\alpha${\in}\ensuremath{\mathcal{A}} (and its matrix representation $M
  (\alpha)$) will be called {\tmstrong{{\tmem{canonical}}}} with respect to
  the isomorphism group {\tmem{G}}, if for $\forall g \in G$:
\end{tmparmod}

Let $A$ be a BIBD with parameters ($v, k, \lambda$) and $\lambda = \lambda
{}_1 + \lambda {}_2 + \ldots + \lambda {}_n$ and $\Lambda = \{ \lambda
\nobracket {}_i$\textbar$1 \leqslant i \leqslant n$\}

As we already know, any CBIBD with parameters \ ($v, k, \{ \lambda {}_1,
\lambda {}_2, \ldots \lambda {}_n \}$) is also an ``ordinary'' BIBD with
parameters \ ($v, k, \lambda {}_1 + \lambda {}_2 + \ldots + \lambda {}_n$). In
other words, if $D$ is a CBIBD represented by the matrix $\mathbb{M} (D)
\tmop{defined} \tmop{by} (3.1), \tmop{then}$

\

\

if we will remove Let's go back to the ``regular'' BIBDs, If information
about \textbar Aut(D)\textbar  is represented as a number $N$, this means that
$N$ is the order of

{\tmem{{\tmem{{\tmstrong{{\tmstrong{Let's go back to the ``regular'' BIBDs}}

\

\

{\tmem{{\tmem{{\tmstrong{Definition 2.1: }}
{\tmstrong{}}}}{\tmstrong{}}}}}}}}}}

\

[1] \begin{itemize}
  
\end{itemize}{\citestar{\label{CITEREFColbournDinitz2007}Colbourn, Charles J.;
Dinitz, Jeffrey H.
(2007),~\href{https://archive.org/details/handbookofcombin0000unse}{\tmtextit{Handbook
of Combinatorial Designs}}~(2nd~ed.), Boca Raton: Chapman \& Hall/
CRC,~\href{https://en.wikipedia.org/wiki/ISBN_(identifier)}{ISBN}~\href{https://en.wikipedia.org/wiki/Special:BookSources/1-58488-506-8}{1-58488-506-8}}}

[2] A.Farad{\v z}ev, Constructive enumeration of combinatorial
objects,~\tmtextit{Probl{\'e}mes combinatoires et th{\'e}orie des graphes},
Orsay 1976, Colloq. int. CNRS No. 260 (1978) pp. 131--135.

[3] A.V.Ivanov, Constructive enumeration of incidence systems,~\tmtextit{Ann.
Discrete Math.}, Vol. 26 (1985) pp. 227--246.

[4] A.Ivanov, Program and some results of constructive enumeration of
incidence systems, \
https://github.com/andrei5055/Incidence-System-Enumeration

[5] R.C.Read, Every one a winner or How to avoid isomorphism searh when
cataloguing combinatorial configurations, {\tmem{Ann. Disrete Math}}. 2
(1978), 107-120.

\

There are 270,474,142 Nonisomorphi 2-(9; 4; 6) Designs Patri R. J. Osterg ‐
{}ard (I do have a copy)

Contrudict my results for 2-(9,4,6)

\

{\"O}sterg{\r a}rd, P.R.J., Kaski, P. Enumeration of 2-(9, 3, {\lambda})
Designs and Their Resolutions.~\tmtextit{Designs, Codes and
Cryptography}~\tmtextbf{27,~}131--137 (2002).
https://doi.org/10.1023/A:1016558720904

\

\begin{tmparmod}{0pt}{0pt}{0tab}%
  \begin{itemize}
    
  \end{itemize}
\end{tmparmod}

\end{document}
